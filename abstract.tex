\begin{enumerate}


The issue of prepayment for MBS (Mortgage-Backed Securities) or other forms of Asset-Backed Securities poses a challenge in terms of modeling. It requires a rigorous approach that involves complex interest rate modeling and consideration of various factors. In this project, we propose an alternative analytical approach to address this challenge, which can be divided into two key parts.\\

Firstly, we aim to reduce the number of existing MBS tickers through clustering techniques. By applying clustering analysis, we can group similar MBS together, leading to a smaller number of larger groups. This approach allows us to analyze the prepayment speed for these groups, providing valuable insights into their consistent behaviors. We anticipate that this clustering analysis will help distinguish potentially four groups exhibiting differing prepayment characteristics.\\

Secondly, we aim to identify statistically significant attributes that can predict prepayment speed. By exploring various factors, including macroeconomic indicators such as interest rates and inflation, we can determine which attributes have the most significant impact on prepayment. Based on this analysis, we will model these attributes as functions to predict prepayment for each group identified in the clustering analysis.\\

By implementing this alternative approach to prepayment modeling, our objective is to streamline the modeling process while still capturing the essential factors that influence prepayment behavior. This will enhance our understanding of prepayment dynamics for small business loan ABS and improve our predictive capabilities.\\

Through this project, we seek to contribute to the advancement of prepayment modeling techniques for small business loan ABS, providing valuable insights to investors and industry professionals. Our goal is to enhance the understanding of prepayment dynamics and improve the accuracy of prepayment predictions for these types of asset-backed securities.\\


\end{enumerate}